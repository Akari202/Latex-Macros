\usepackage{fancyhdr}
\usepackage{extramarks}
\usepackage{amsmath}
\usepackage{amsthm}
\usepackage{amsfonts}
\usepackage{mathtools}
\usepackage{gensymb}
\usepackage{tikz}
\usepackage[plain]{algorithm}
\usepackage{algpseudocode}
\usepackage{placeins}
\usepackage[outputdir=../]{minted}
\usepackage{caption}
\usepackage{luacode}
\usepackage{mathrsfs}
% https://tex.stackexchange.com/questions/2291/how-do-i-change-the-enumerate-list-format-to-use-letters-instead-of-the-defaul
% \usepackage{enumitem}
\usepackage[shortlabels]{enumitem}
\usepackage{setspace}

\usetikzlibrary{automata}
\usetikzlibrary{positioning}
\usetikzlibrary{patterns}
\usetikzlibrary{calc}
\usetikzlibrary{decorations.pathmorphing}
\usetikzlibrary{decorations.markings}

\def\therefore{\boldsymbol{\text{ }
\leavevmode
\lower0.4ex\hbox{$\cdot$}
\kern-.5em\raise0.7ex\hbox{$\cdot$}
\kern-0.55em\lower0.4ex\hbox{$\cdot$}
\thinspace\text{ }}}

\newcommand{\alg}[1]{\textsc{\bfseries \footnotesize #1}}
% \newcommand{\derivx}[1]{\frac{\mathrm{d}}{\mathrm{d}x} (#1)}
\newcommand{\derivp}[3]{\frac{\mathrm{d} #1}{\mathrm{d} #2} \left\lparen#3\right\rparen}
\newcommand{\pderivp}[3]{\frac{\partial #1}{\partial #2} \left\lparen#3\right\rparen}
\newcommand{\deriv}[2]{\frac{\mathrm{d} #1}{\mathrm{d} #2}}
\newcommand{\pderiv}[2]{\frac{\partial #1}{\partial #2}}
\newcommand{\indefinteg}[2]{\int #1 \, \mathrm{d} #2}
\newcommand{\evalat}[2]{\left. #2 \right\rvert_{#1}}
\newcommand{\evalover}[3]{\left. #3 \right\rvert_{#1}^{#2}}
\newcommand{\dx}[1]{\mathrm{d} #1}
\newcommand{\solution}{\textbf{\large Solution:}}
% \newcommand{\solution}{\textbf{\large Solution:}\pagebreak}
\newcommand{\step}[1]{H(#1)}
\newcommand{\dirac}[1]{\delta(#1)}
\newcommand{\laplace}[1]{\mathscr{L}\left\lbrace #1 \right\rbrace}
\newcommand{\ilaplace}[1]{\mathscr{L}^{-1}\left\lbrace #1 \right\rbrace}
\newcommand{\limit}[3]{\lim_{#1 \to #2}\left\lparen #3 \right\rparen}
% \newcommand{\E}{\mathrm{E}}
% \newcommand{\Var}{\mathrm{Var}}
% \newcommand{\Cov}{\mathrm{Cov}}
% \newcommand{\Bias}{\mathrm{Bias}}

% https://tex.stackexchange.com/questions/289949/defining-tg-so-it-behaves-exactly-like-sin-and-cos
\DeclareMathOperator{\sgn}{sgn}

% https://tex.stackexchange.com/questions/254044/caption-and-label-on-minted-code
\newenvironment{code}{\captionsetup{type=listing}}{}
% \SetupFloatingEnvironment{listing}{name=Source Code}

\newenvironment{hint}
    {\begin{quote}\small\itshape \textbf{Hint:} \setstretch{1.5}}
    {\end{quote}}

% https://tex.stackexchange.com/questions/538603/automatically-add-parentheses-to-cos-sin-log-ecc-allowing-subscripts
\begin{luacode}
function add_parens ( s )
  ll= { "sin" , "cos" , "tan" , "cot", "sgn" }
  for i=1,#ll do
    s = s:gsub ( "(\\"..ll[i]..")%s+(%a+)"   , "%1(%2)" ) -- \sin x
    s = s:gsub ( "(\\"..ll[i]..")%s-(\\%a+)" , "%1(%2)" ) -- \cos\alpha
    s = s:gsub ( "(\\"..ll[i]..")%s-(%b{})"  , "%1(%2)" ) -- \tan{y}
    s = s:gsub ( "(\\"..ll[i]..")%s-(^%s-%w)%s-(\\?%a+)"  , "%1%2(%3)" ) -- \sin^2 z 
    s = s:gsub ( "(\\"..ll[i]..")%s-(^%s-%w)%s-(%b{})"    , "%1%2(%3)" ) -- \sin^3 {u}
    s = s:gsub ( "(\\"..ll[i]..")%s-(^%s-%b{})%s-(\\?%a+)", "%1%2(%3)" ) -- \cos ^{2} \pi
    s = s:gsub ( "(\\"..ll[i]..")%s-(^%s-%b{})%s-(%b{})"  , "%1%2(%3)" ) -- \cos ^{10} {v}
  end  
  return s
end
\end{luacode}
\newcommand\AddParensOn{\directlua{ luatexbase.add_to_callback 
  ( "process_input_buffer" , add_parens , "AddParens" )}}
\newcommand\AddParensOff{\directlua{ luatexbase.remove_from_callback 
  ( "process_input_buffer" , "AddParens" )}}